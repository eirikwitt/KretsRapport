\section{Teori}
  \subsection{Tallrepresentasjon}
    Det er 3 vanlige måter å representere heltall binært: Magnitude uten fortegn, magnitude med fortegn,
    og toerkomplement.
    I denne laboppgaven har tallene blitt representert med toerkompliment.
    På toerkomplimentsform representeres positive tall som magnitude uten fortegn.
    Negative tall representeres ved å ta inversen til absoluttveriden av tallet, og legge til 1.
    Å ta inversen til et binærtall gjøres ved å gjøre om alle 1-ere til 0, og alle 0-ere til 1.
    Man kan se om et tall er negativt ved å se på det MSB (det mest signifikante bittet). Hvis det er 1 er tallet negativt,
    hvis det er 0 er tallet positivt.
  \subsection{Absoluttverdi}
    En absoluttverdikrets skal ta inn et tall, og gi ut absoluttverdien av tallet
    Å ta absoluttverdien av et tall på toerkomplementsform gjøres i 3 steg.
    \begin{enumerate}
      \item Sjekke fortegnet til tallet ved å se på MSB.
      \begin{itemize}
        \item Hvis tallet er positivt, gi talet som output.
        \item Hvis tallet er negativt, utfør punkt 2 og 3.
      \end{itemize}
      \item Ta inversen av tallet.
      \item Legg til 1 til tallet og gi det som output.
    \end{enumerate}
  \subsection{Modifisert Ripple Carry adder}
    En vanlig ripple carry adder består av en serie med heladdere som utfører addisjon på bitnivå, og sender ut et summ-bit til output-bussen, og et carry bit som går videre til den nesteheladderen. En slik adder tar inn 2 tall på toerkomplementsform eller som magnitude uten fortegns, og legger dem sammen. Den Modifiserte ripple carry adderen som har blitt brukt i denne laboppgaven skal derimot ta inn
    ett tall på toerkomplimentsform og et bit. Det den skal gi ut er tallet plus det ene bittet.
    Adderen er bygget opp av halvaddere. En halvadder har 2 inputs, (A og B), og 2 outputs, (S og C).
    Halvadderen legger sammen de to bitsene, og sender signal på S hvis summen blir 1, og signal på C hvis summen blir 2.
    I den modifiserte ripple carry adderen er halvadderene koblet sammen koblet sammen slik at et av carryen fra det forrige bittet går til et av inputtene på det neste bittet.
  \subsection{Tidsforsinkelse og kritisk sti}
