\usepackage[utf8]{inputenc}
\usepackage[T1]{fontenc}
\usepackage{siunitx}

% Babel definerer standard autogenerert tekst for mange forskjellige språk.
% Når vi senere kommer til figurtekst og lignende, vil dere se at Babel forsikrer
% at det står Figur og ikke Figure i den genererte figurteksten. Det samme gjelder
% for innholdsfortegnelse etc.
\usepackage[norsk]{babel}

% << >> erstatter "" i referanseliste (Frivillig)
\usepackage{csquotes}

% Gjør at output PDF støtter linker.
\usepackage{hyperref}

% For mer fleksible nummererte lister.
\usepackage{enumitem}

% For å inkludere bilder i rapporten. Takler ganske mange formater
\usepackage{graphicx}
% Søkepath for å finne bilder. Dette er alternativt til å skrive full path der dere legger inn bildet.
\graphicspath{{./Bilder/}}
\usepackage{subcaption}

% Brukes for å få tekst til å gå over flere rader i en tabell.
\usepackage{multirow}

\usepackage{amsmath}
\usepackage{amssymb}

\usepackage{tikz}
\usepackage{circuitikz}

\usepackage[lmargin=25mm,rmargin=25mm,tmargin=27mm,bmargin=30mm]{geometry}

% % En av pakkene som kan brukes til referanser
\usepackage[style=ieee, citestyle=numeric-comp]{biblatex}
\addbibresource{mylib.bib}
